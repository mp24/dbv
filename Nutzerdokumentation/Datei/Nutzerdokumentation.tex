\documentclass[a4paper,11pt,parskip]{article}

\usepackage[ngerman]{babel}
\usepackage[T1]{fontenc}
\usepackage[utf8]{inputenc}
\usepackage{graphics}
\usepackage{amsmath}
\usepackage{blindtext}


\begin{document}




\title{Nutzerdokumentation im Modul Digitale Bildverabeitung\\  
\textbf{Detektion und Erkennung von QR-Verkehrszeichen}}
\author{Tina Neumann und Minh Pham}

\maketitle

Mit dem vorliegenden ImageJ-PlugIn ist es möglich, QR-Codes im Bild zu erkennen und diese mit jeweils ihren Verkehrzeichen zuordnen zu können.\\

\section{Programmfunktion}

\textbf{Eingabe}: RGB-Bild (1.Option: Bild im ImageJ geöffnet; 2.Option: Aufruf über Pfadparameter)\\ \\
\textbf{Ausgabe}: (ggf. leere) Liste von Bounding-Boxen alles im Bild erkannten QR-Verkehrszeichen, je mit Verkehrszeichen-Typ
\textbf{Anzeige}: Bounding-Boxen mit Typsymbol im Overlay
chen-Typ\\ \\
\textbf{Aktueller Stand}: Anzeige Boxen als Overlay(Boxennummer), weitere Bearbeitung folgen: Erkennung von QR-Verkehrzeichen

\newpage
\title{Projektdokumentation im Modul Digitale Bildverabeitung\\
\textbf{Detektion und Erkennung von QR-Verkehrszeichen}

\maketitle
\section{Vorgehensweise}
\subsection{erster Ansatz}
Der erste Ansatz basiert auf das Auffinden von Linien, dabei benutzen wir die Funktionen Find Edges und MakeBinary. Das Ziel ist es, die QR-Codes anhand der Linien entlang der Boxen zu finden.

\subsection{zweiter Ansatz}
Der zweite Ansatz basiert auf das Suchen von QR-Codes in Abhängikeit von Verkehrsschildern. Dabei suchen wir zunächst nach roten und blauen Regionen im Bild, welche die Verkehrsschilder auffinden soll. Anhand dieser wird der darunter liegende QR-Code detektiert.
\end{document}