\documentclass[a4paper,11pt,parskip]{article}

\usepackage[ngerman]{babel}
\usepackage[T1]{fontenc}
\usepackage[utf8]{inputenc}
\usepackage{graphics}
\usepackage{amsmath}
\usepackage{blindtext}


\begin{document}




\title{Nutzerdokumentation im Modul Digitale Bildverabeitung\\  
\textbf{Detektion und Erkennung von QR-Verkehrszeichen}}
\author{Tina Neumann und Minh Pham}

\maketitle

Mit dem vorliegenden ImageJ-PlugIn ist es möglich, QR-Codes in einem RGB Bild zu erkennen.\\

\section{Programmfunktion}

\textbf{Eingabe}: RGB-Bild (1.Option: Bild im ImageJ geöffnet; 2.Option: Aufruf über Pfadparameter)\\ \\
\textbf{Ausgabe}: (ggf. leere) Liste von Bounding-Boxen alles im Bild erkannten QR-Verkehrszeichen, je mit Verkehrszeichen-Typ
\textbf{Anzeige}: Bounding-Boxen mit Typsymbol im Overlay
chen-Typ\\ \\
\textbf{Aktueller Stand}: Anzeige Boxen als Overlay(Boxennummer), weitere Bearbeitung folgen: Erkennung von QR-Verkehrzeichen

\section{Nutzung}
Es gibt zwei Möglichkeiten, das PlugIn zu starten. Zum einen wird das PlugIn über dem Makro gestartet. In dem Makro befindet sich der übergebene Pfad, in dem das Bild liegt. Zum anderen kann man im ImageJ ein Bild auswählen und das PlugIn aufrufen, welche erstellt wurde.

\end{document}